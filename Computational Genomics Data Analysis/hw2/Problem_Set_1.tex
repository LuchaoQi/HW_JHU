\documentclass[11pt]{article}
%https://www.gradescope.com/help#help-center-item-answer-formatting-guide
\usepackage{graphicx}
\usepackage{wrapfig}
\usepackage{url}
\usepackage{wrapfig}
\usepackage{color}
\usepackage{marvosym}
\usepackage{enumerate}
\usepackage{subfigure}
\usepackage{tikz}
\usepackage[fleqn]{amsmath}
\usepackage{amssymb}
\usepackage{hyperref} 
\usepackage[many]{tcolorbox}
\usepackage{lipsum}
\usepackage{float}
\usepackage{trimclip}
\usepackage{listings}
\usepackage{environ}% http://ctan.org/pkg/environ
\usepackage{wasysym}
\usepackage{array}


\oddsidemargin 0mm
\evensidemargin 5mm
\topmargin -20mm
\textheight 240mm
\textwidth 160mm

\bgroup
\def\arraystretch{1.5}
\newcolumntype{x}[1]{>{\centering\arraybackslash\hspace{0pt}}p{#1}}
\newcolumntype{z}[1]{>{\centering\arraybackslash}m{#1}}



%Arguments are 1 - height, 2 - box title
\newtcolorbox{textanswerbox}[2]{%
 width=\textwidth,colback=white,colframe=blue!30!black,floatplacement=H,height=#1,title=#2,clip lower=true,before upper={\parindent0em}}
 
 \newtcolorbox{eqanswerbox}[1]{%
 width=#1,colback=white,colframe=black,floatplacement=H,height=3em,sharp corners=all,clip lower=true,before upper={\parindent0em}}
 
 %Arguments are 1 - height, 2 - box title
 \NewEnviron{answertext}[2]{
 	\noindent
	\marginbox*{0pt 10pt}{
  	\clipbox{0pt 0pt 0pt 0pt}{
	\begin{textanswerbox}{#1}{#2}
	\BODY
	\end{textanswerbox}
	}
	}
}

%Arguments are 1 - height, 2 - box title, 3 - column definition
 \NewEnviron{answertable}[3]{
 	\noindent
	\marginbox*{0pt 10pt}{
  	\clipbox{0pt 0pt 0pt 0pt}{
	\begin{textanswerbox}{#1}{#2}
		\vspace{-0.5cm}
        		\begin{table}[H]
        		\centering 
        		\begin{tabular}{#3}
        			\BODY
        		\end{tabular}
        		\end{table}
	\end{textanswerbox}
	}
	}
}

 %Arguments are 1 - height, 2 - box title, 3 - title, 4- equation label, 5 - equation box width
 \NewEnviron{answerequation}[5]{
 	\noindent
	\marginbox*{0pt 10pt}{
  	\clipbox{0pt 0pt 0pt 0pt}{
	\begin{textanswerbox}{#1}{#2}
		\vspace{-0.5cm}
        		\begin{table}[H]
        		\centering 
		\renewcommand{\arraystretch}{0.5}% Tighter

        		\begin{tabular}{#3}
        			#4 =	&
		  	\clipbox{0pt 0pt 0pt 0pt}{

			\begin{eqanswerbox}{#5}
				$\BODY$
			\end{eqanswerbox}
			} \\ 
        		\end{tabular}
        		\end{table}
		
	\end{textanswerbox}
	}
	}
}

 %Arguments are 1 - height, 2 - box title
 \NewEnviron{answerderivation}[2]{
 	\noindent
	\marginbox*{0pt 10pt}{
  	\clipbox{0pt 0pt 0pt 0pt}{
	\begin{textanswerbox}{#1}{#2}
	\BODY
	\end{textanswerbox}
	}
	}
}

\newcommand{\vwi}{{\bf w}_i}
\newcommand{\vw}{{\bf w}}
\newcommand{\vx}{{\bf x}}
\newcommand{\vy}{{\bf y}}
\newcommand{\vxi}{{\bf x}_i}
\newcommand{\yi}{y_i}
\newcommand{\vxj}{{\bf x}_j}
\newcommand{\vxn}{{\bf x}_n}
\newcommand{\yj}{y_j}
\newcommand{\ai}{\alpha_i}
\newcommand{\aj}{\alpha_j}
\newcommand{\X}{{\bf X}}
\newcommand{\Y}{{\bf Y}}
\newcommand{\vz}{{\bf z}}
\newcommand{\msigma}{{\bf \Sigma}}
\newcommand{\vmu}{{\bf \mu}}
\newcommand{\vmuk}{{\bf \mu}_k}
\newcommand{\msigmak}{{\bf \Sigma}_k}
\newcommand{\vmuj}{{\bf \mu}_j}
\newcommand{\msigmaj}{{\bf \Sigma}_j}
\newcommand{\pij}{\pi_j}
\newcommand{\pik}{\pi_k}
\newcommand{\D}{\mathcal{D}}
\newcommand{\el}{\mathcal{L}}
\newcommand{\N}{\mathcal{N}}
\newcommand{\vxij}{{\bf x}_{ij}}
\newcommand{\vt}{{\bf t}}
\newcommand{\yh}{\hat{y}}
\newcommand{\code}[1]{{\footnotesize \tt #1}}
\newcommand{\alphai}{\alpha_i}

\newcommand{\Checked}{{\LARGE \XBox}}%
\newcommand{\Unchecked}{{\LARGE \Square}}%
\newcommand{\TextRequired}{{\textbf{Place Answer Here}}}%
\newcommand{\EquationRequired}{\textbf{Type Equation Here}}%


\newcommand{\answertextheight}{5cm}
\newcommand{\answertableheight}{4cm}
\newcommand{\answerequationheight}{2.5cm}
\newcommand{\answerderivationheight}{14cm}

\newcounter{QuestionCounter}
\newcounter{SubQuestionCounter}[QuestionCounter]
\setcounter{SubQuestionCounter}{1}

\newcommand{\subquestiontitle}{Question~}
\newcommand{\newquestion}{\stepcounter{QuestionCounter}\setcounter{SubQuestionCounter}{1}\newpage}
\newcommand{\newsubquestion}{\stepcounter{SubQuestionCounter}}


\lstset{language=[LaTeX]TeX,basicstyle=\ttfamily\bf}

\pagestyle{myheadings}
\markboth{Problem set 0}{Spring 2019 EN.601.448/648 Computational genomics: Problem set 1}


\title{EN.601.448/648 Computational genomics: Problem set 1}
\author{YOUR\_NAME (YOUR\_JHED)} 
\date{} 



\begin{document}
\maketitle
\thispagestyle{headings}


\newquestion

\section*{\arabic{QuestionCounter}. Central Dogma (9 points) }
{

\renewcommand{\answertextheight}{4cm}
\begin{answertext}{\answertextheight}{\subquestiontitle a }
RNA sequence: \TextRequired \\
AA sequence: \TextRequired
\end{answertext}
\newsubquestion

\begin{answertext}{\answertextheight}{\subquestiontitle b (i) }
DNA sequence: \TextRequired \\
AA sequence: \TextRequired
\end{answertext}
\newsubquestion

\begin{answertext}{\answertextheight}{\subquestiontitle b (ii) }
DNA sequence: \TextRequired \\
AA sequence: \TextRequired
\end{answertext}
\newsubquestion

\begin{answertext}{\answertextheight}{\subquestiontitle b (iii) }
DNA sequence: \TextRequired \\
AA sequence: \TextRequired
\end{answertext}
\newsubquestion

\begin{answertext}{\answertextheight}{\subquestiontitle b (iv) }
DNA sequence: \TextRequired \\
AA sequence: \TextRequired
\end{answertext}
\newsubquestion
\renewcommand{\answertextheight}{2cm}
\begin{answertext}{\answertextheight}{\subquestiontitle c (i) }
chromosome: \TextRequired \\
approximate position: \TextRequired \\
\end{answertext}
\newsubquestion

\begin{answertable}{2.5cm}{\subquestiontitle c(ii)}{x{0.5cm}p{5cm}}
\Unchecked &  (a) Intronic \\ 
\Unchecked & (b) Exonic \\     
\end{answertable}
\newsubquestion


\begin{answertable}{3.8cm}{\subquestiontitle c(iii)}{x{0.5cm}p{5cm}}
\Unchecked &  (a) non-synonymous \\ 
\Unchecked & (b) synonymous\\     
\Unchecked & (c) frameshift
\end{answertable}
\newsubquestion


}

\newquestion

\section*{\arabic{QuestionCounter}. Genetic Variation in the human genome (6 points) }
{


\renewcommand{\answertextheight}{2cm}
\begin{answertable}{2.5cm}{\subquestiontitle (a)}{x{0.5cm}p{5cm}}
\Unchecked &  True \\ 
\Unchecked & False \\    
\end{answertable}
\newsubquestion

\renewcommand{\answertextheight}{2cm}
\begin{answertable}{2.5cm}{\subquestiontitle (b)}{x{0.5cm}p{5cm}}
\Unchecked &  True \\ 
\Unchecked & False \\    
\end{answertable}
\newsubquestion

\renewcommand{\answertextheight}{2cm}
\begin{answertable}{2.5cm}{\subquestiontitle (c)}{x{0.5cm}p{5cm}}
\Unchecked &  True \\ 
\Unchecked & False \\    
\end{answertable}
\newsubquestion

\renewcommand{\answertextheight}{2cm}
\begin{answertable}{2.5cm}{\subquestiontitle (d)}{x{0.5cm}p{5cm}}
\Unchecked &  True \\ 
\Unchecked & False \\    
\end{answertable}
\newsubquestion

\renewcommand{\answertextheight}{2cm}
\begin{answertable}{2.5cm}{\subquestiontitle (e)}{x{0.5cm}p{5cm}}
\Unchecked &  True \\ 
\Unchecked & False \\    
\end{answertable}
\newsubquestion

\renewcommand{\answertextheight}{2cm}
\begin{answertable}{2.5cm}{\subquestiontitle (f)}{x{0.5cm}p{5cm}}
\Unchecked &  True \\ 
\Unchecked & False \\    
\end{answertable}
\newsubquestion


}

\newquestion



\section*{\arabic{QuestionCounter}. Modeling genotype effects (15 points)}
{
\renewcommand{\answertextheight}{10cm}
\begin{answertext}{\answertextheight}{\subquestiontitle a: Three outcomes for AT}
Outcome 1: \TextRequired \\
Data encoding:  \TextRequired \\
Model and description: \TextRequired \\

Outcome 2: \TextRequired \\
Data encoding:  \TextRequired \\
Model and description: \TextRequired \\

Outcome 3: \TextRequired \\
Data encoding:  \TextRequired \\
Model and description: \TextRequired \\


\end{answertext}
\newsubquestion

\renewcommand{\answertextheight}{6cm}
\begin{answertext}{\answertextheight}{\subquestiontitle b}
\TextRequired
\end{answertext}
\newsubquestion

\renewcommand{\answertextheight}{8cm}
\begin{answertext}{\answertextheight}{\subquestiontitle c: Nonlinear interaction}
\TextRequired
\end{answertext}

}

\newquestion

\section*{\arabic{QuestionCounter}. Maximum Likelihood estimation}
{

\renewcommand{\answertextheight}{2cm}
\begin{answertext}{\answertextheight}{\subquestiontitle a: correlation}
correlation = \TextRequired  \\
two-tailed pvalue = \TextRequired \\
\end{answertext}
\newsubquestion


\renewcommand{\answertextheight}{5cm}
\begin{answertext}{\answertextheight}{\subquestiontitle b: bivariate normal}
$\mu=$
$$
\begin{bmatrix}
    \TextRequired      & \TextRequired
\end{bmatrix}
$$
$\Sigma=$
$$
\begin{bmatrix}
    \TextRequired      & \TextRequired\\
    \TextRequired       & \TextRequired
\end{bmatrix}
$$


\end{answertext}
\newsubquestion


\renewcommand{\answertextheight}{3cm}
\begin{answertext}{\answertextheight}{\subquestiontitle c: Expected value}
$E[TNNT2 | SOX2]=$ \TextRequired
\end{answertext}

}
\newquestion



\section*{\arabic{QuestionCounter}. Logistic Regression (15 points)}
{
\renewcommand{\answertextheight}{2cm}
\begin{answertext}{\answertextheight}{\subquestiontitle a}
precision = \TextRequired  \\
recall = \TextRequired
\end{answertext}
\newsubquestion

\renewcommand{\answertextheight}{2cm}
\begin{answertext}{\answertextheight}{\subquestiontitle b}
precision = \TextRequired  \\
recall = \TextRequired
\end{answertext}
\newsubquestion

\renewcommand{\answertextheight}{6cm}
\begin{answertext}{\answertextheight}{\subquestiontitle c: Performance differences}
\TextRequired 
\end{answertext}
\newsubquestion

\begin{answertext}{\answertextheight}{\subquestiontitle d: How to better select the 10 genes?}
\TextRequired 
\end{answertext}
\newquestion


}





\section*{\arabic{QuestionCounter}. Genome-wide association studies (30 points)}
{
\renewcommand{\answertextheight}{3cm}
\begin{answertext}{\answertextheight}{\subquestiontitle a}
Number of SNPs with MAF $>0.03$ = \TextRequired  \\
Number of SNPs with MAF $>0.05$ = \TextRequired  \\
Number of SNPs with MAF $>0.1$ = \TextRequired  
\end{answertext}
\newsubquestion

\renewcommand{\answertextheight}{4cm}
\begin{answertext}{\answertextheight}{\subquestiontitle c: Interpretation of the logistic regression parameter}
\TextRequired 
\end{answertext}
\newsubquestion


\renewcommand{\answertextheight}{10cm}
\begin{answertext}{\answertextheight}{\subquestiontitle d: How to better select the 10 genes?}
regression parameter $\beta$ for SNP 10 = \TextRequired \\
Plot of MLE for $\beta$ and values near it. 
\begin{center}
   \includegraphics[scale=0.3]{plotRequired.png}
\end{center}
\end{answertext}


\renewcommand{\answertextheight}{6cm}
\begin{answertext}{\answertextheight}{\subquestiontitle e: Confounding factors}
\TextRequired 
\end{answertext}
\newquestion


}



\section*{\arabic{QuestionCounter}. Ridge regression (10 points)}
{
\renewcommand{\answertextheight}{4cm}
\begin{answertext}{\answertextheight}{\subquestiontitle a}
Optimal $\alpha$ = \TextRequired \\
Justification: \TextRequired
\end{answertext}
\newsubquestion
}
\end{document}
